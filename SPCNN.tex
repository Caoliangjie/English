\documentclass[10pt,twocolumn,letterpaper]{article}
\usepackage{cvpr}
\usepackage{times}
\usepackage{epsfig}
\usepackage{graphicx}
\usepackage{amsmath}
\usepackage{amssymb}
\usepackage[breaklinks=true,bookmarks=false,backref=page]{hyperref}
\cvprfinalcopy
\def\cvprPaperID{****}
\def\httilde{\mbox{\tt\raisebox{-.5ex}{\symbol{126}}}}
\begin{document}
\title{\textbf{Spherical CNNS
}}
\author{Liangjie Cao\\\\ July 6, 2018}
\maketitle
\abstract{Convolutional Neural Networks (CNNs) have become the method of choice for
learning problems involving 2D planar images. However, a number of problems of
recent interest have created a demand for models that can analyze spherical images.
Examples include omnidirectional vision for drones, robots, and autonomous cars,
molecular regression problems, and global weather and climate modelling. A
naive application of convolutional networks to a planar projection of the spherical
signal is destined to fail, because the space-varying distortions introduced by such
a projection will make translational weight sharing ineffective.
In this paper the authors introduce the building blocks for constructing spherical CNNs.
They propose a definition for the spherical cross-correlation that is both expres-
sive and rotation-equivariant. The spherical correlation satisfies a generalized
Fourier theorem, which allows us to compute it efficiently using a generalized
(non-commutative) Fast Fourier Transform (FFT) algorithm. They demonstrate the
computational efficiency, numerical accuracy, and effectiveness of spherical CNNs
applied to 3D model recognition and atomization energy regression.}
\section{Introduction}
   \begin{figure}[!htb]
 \centering
 \includegraphics[width=2in]{planer.png}\\
 \caption{Any planar projection of a spherical signal will result in distortions. Rotation of a spherical signal cannot be emulated by translation of its planar projection.}\label{Figure1}
 \end{figure}
Convolutional networks are able to detect local patterns regardless
of their position in the image. Like patterns in a planar image,
patterns on the sphere can move around, but in this case the ``move"
is a 3D rotation instead of a translation. In analogy to the planar
CNN, we would like to build a network that can detect patterns
regardless of how they are rotated over the sphere.
\par As shown in Figure~\ref{Figure1}, there is no good way to use translational
convolution or cross-correlation\footnote{Despite the name, CNNs typically use cross-correlation instead of convolution in the forward pass. In this paper the authors  generally use the term cross-correlation, or correlation for short.}to analyze spherical signals. The
most obvious approach, then, is to change the definition of cross-
correlation by replacing filter translations by rotations. Doing so,
they run into a subtle but important difference between the plane
and the sphere: whereas the space of moves for the plane (2D
translations) is itself isomorphic to the plane, the space of moves
for the sphere (3D rotations) is a different, three-dimensional
manifold called SO(3)\footnote{To be more precise: although the symmetry group of the plane contains more than just translations, the translations form a subgroup that acts on the plane. In the case of the sphere there is no coherent way to define a composition for points on the sphere, and so the sphere cannot act on itself (it is not a group).}. It follows that the result of a spherical
correlation (the output feature map) is to be considered a signal on $SO(3)$, not a signal on the sphere, $S^2$. For this reason, they deploy $SO(3)$ group correlation in the higher layers of a spherical CNN~\cite{name1}. Then they address both of these problems using techniques from non-commutative harmonic analysis~\cite{name2,name3}. 
\section{Correlation On The Sphere And Rotation Group}
They will explain the $S^2$ and $SO(3)$ correlation by analogy to the classical planar $\mathbb{Z}^2$ correlation. The planar correlation means the value of the output feature map at translation $x\in\mathbb{Z}^2$ is computed as an inner product between the input feature map and a filter, shifted by $x$. And the spherical correlation means the value of the output feature map evaluated at rotation $R\in{SO(3)}$ is computed as an inner product between the input feature map and a filter, rotated by $R$.
\par In order to define the spherical correlation, the authors need to know not only how to rotate points $x\in{S^2}$ but also how to rotate filters (\emph{i.e.} functions) on the sphere. To this end, they introduce the rotation operator $L_R$ that takes a function $f$ and produces a rotated function $L_R{f}$ by composing $f$ with the rotation $R^{-1}$:
\begin{equation}
[L_R{f}](x)=f(R^{-1}x)
\end{equation}\label{2}
Due to the inverse on $R$, they have $L_{RR'}=L_{R}L_{R'}$.
\par The inner product on the vector space of spherical signals is defined as:
\begin{equation}
<\psi,f>=\int_{S^2}\sum_{k=1}^{K}\psi_{k}(x)f_k(x)dx
\end{equation}\label{Eq3}
The integration measure $dx$ denotes the standard rotation invariant integration measure on the sphere, which can be expressed as $d\alpha$ $R$ $sin(\beta)d\beta/4\pi$ in R spherical coordinates. The invariance of the measure ensures that $\int_{S^2}f(Rx)dx=\int_{S^2}f(x)dx$, for any rotation $R\in{SO(3)}$. That is, the volume under a spherical height map does not change when rotated. Using this fact, they can show that $L_{R^{-1}}$ is adjoint to $L_R$, which implies that $L_R$ is unitary:
\begin{equation}
<L_R{\psi},f>=\int_{S^2}\sum_{k=1}^{K}\psi_k(R^{-1}x)f_{k}(x)dx
\end{equation}\label{E}
Then we know that this two function is equal.
\bibliographystyle{abbrv}
\bibliography{yinyong8}
\end{document}