\documentclass[10pt,twocolumn,letterpaper]{article}
\usepackage{cvpr}
\usepackage{times}
\usepackage{epsfig}
\usepackage{graphicx}
\usepackage{amsmath}
\usepackage{amssymb}
\usepackage{array}
\usepackage[breaklinks=true,bookmarks=false,backref=page]{hyperref}
\cvprfinalcopy
\def\cvprPaperID{****}
\def\httilde{\mbox{\tt\raisebox{-.5ex}{\symbol{126}}}}
\begin{document}
\title{\textbf{Image Question Answering using Convolutional Neural Network with Dynamic Parameter Prediction II
}}
\author{Liangjie Cao\\\\ Jun 28, 2018}
\maketitle
\section{Final datasets}
The authors evaluate the proposed network on several public ImageQA benchmark datasets such as DAQUAR~\cite{name17}, COCOQA~\cite{name23} and VQA~\cite{name1}. They collected question-answer pairs from existing image datasets and most of the answers are single words or short phrases.
\par DAQUAR is based on NYUDv2 dataset, and pro-
vides two benchmarks. DAQUAR-all consists of 6,795 and
5,673 questions for training and testing respectively, and
includes 894 categories in answer. DAQUAR-reduced in-
cludes only 37 answer categories for 3,876 training and 297
testing questions. Some questions in this dataset are associ-
ated with a set of multiple answers.
 \par VQA~\cite{name1}, which is also based on MS COCO dataset~\cite{name15},
contains the largest number of questions: 248,349 for training, 121,512 for validation, and 244,302 for testing, where the testing data is split into test-dev, test-standard, test-challenge and test-reserve. Each question is associated with 10 answers annotated by different people. About 90\% of answers have single words and 98\% of answers do not exceed three words.
  \begin{table}[h]
  \resizebox{0.47\textwidth}{30mm}{
 \begin{tabular}{p{1.2cm}|p{0.5cm}p{0.5cm}p{0.5cm}p{0.7cm}|p{0.5cm}p{0.5cm}p{0.5cm}p{0.5cm}}
        & \multicolumn{4}{c|}{Open-Ended} & \multicolumn{4}{c}{Multiple-Choice}  \\
     &All & Y/N & Num & Others & All&Y/N&Num&Others \\
    \hline
    Question & 48.09 & 75.66 & 36.70 & \small{27.14}& 53.68 &75.71 & 37.05 & 38.64 \\
     Image & 28.13 & 64.01 & 00.42 & 03.77 & 30.53 &69.87 & 00.45 & 03.76\\
    Q+I & 52.64 & 75.55 & 33.67 & 37.37  & 58.97 & 75.59 & 34.35 & 50.33\\
    LSTM Q & 48.76 & 78.20 & 35.68 & 26.59 & 54.75 & 78.22 & 36.82 & 38.78\\
     LSTM Q+I &53.74& 78.94 & 35.24 & 36.42 & 57.17 & 78.95 & 35.80 & 43.41  \\
     \hline
     CONCAT & 54.70 & 77.09 & 36.62 & 39.67& 59.92 & 77.10 & 37.48 & 50.31\\
     RAND-GRU & 55.46 & 79.58 & 36.20 & 39.23& 61.18 & 79.64 & 38.07 & 50.36\\
     CNN-FIXED & 56.74 & 80.48 & 37.20 & 40.90& 61.95 & 80.56 & 38.32 & 51.40\\
     DPPnet & 57.22 & 80.71 & 37.24 & 41.69& 62.48 & 80.79 & 38.94 & 52.16\\
     \hline
  \end{tabular}}
       \caption{\large{performances of different methods on Shanghaitech dataset}} \label{Table1}
  \end{table}
\section{Evaluation Metrics}
VQA dataset provides open-ended task and multiple-choice task for evaluation. For open-ended task, the answer can be any word or phrase while an answer should be chosen out of 18 candidate answers in the multiple-choice task.
In both cases, answers are evaluated by accuracy reflecting
human consensus. For predicted answer $a_i$ and target an-
swer set $T_i$ of the $i^{th}$ example, the accuracy is given by
\begin{equation}
Acc_{VQA}=\frac{1}{N}\sum_{i=1}^{N}\min\{\frac{\Sigma_{t\in{T^i}}\Pi[a_i=t]}{3},1\}
\end{equation}
where $\Pi[\cdot]$ denotes an indicator function. In other words, a
predicted answer is regarded as a correct one if at least three
annotators agree, and the score depends on the number of
agreements if the predicted answer is not correct.
\section{Results}
The authors test three independent datasets, VQA, COCO-QA,
and DAQUAR, and first present the results for VQA dataset
in Table~\ref{Table1}. The proposed Dynamic Parameter Prediction
network (DPPnet) outperforms all existing methods non-
trivially. We performed controlled experiments to ana-
lyze the contribution of individual components in the pro-
posed algorithm—dynamic parameter prediction, use of
pre-trained GRU and CNN fine-tuning, and trained 3 addi-
tional models, CONCAT, RAND-GRU, and CNN-FIXED.
  \begin{figure*}[!ht]
 \centering
 \includegraphics[width=0.8\textwidth]{fi.png}\\
 \caption{\textbf{Sample images and questions in VQA dataset~\cite{name1}. Each question requires a different type and/or level of understanding of the
corresponding input image to find correct answer. Answers in blue are correct while answers in red are incorrect. For the incorrect answers,
ground-truth answers are provided within the parentheses}}\label{Figure1}
 \end{figure*}
The qualitative results of the proposed algorithm are pre-
sented in Figure~\ref{Figure1}. In general, the proposed network is successful to handle various types of questions that need different levels of semantic understanding. Figure~\ref{Figure1}(a) shows that
the network is able to adapt recognition tasks depending on
questions. However, it often fails in the questions asking the
number of occurrences since these questions involve the dif-
ficult tasks (\emph{e.g.}, object detection) to learn only with image level annotations. On the other hand, the proposed network
is effective to find the answers for the same question on dif-
ferent images fairly well as illustrated in Figure~\ref{Figure1}(b). Refer to our project website 2 for more detailed results.
\bibliographystyle{abbrv}
\bibliography{yinyong6}
\end{document}

