\documentclass[50pt]{article}
\usepackage{ctex}
\usepackage{CJK}
\setlength{\parindent}{2em}
\begin{document}
\title{Hyperloop}
\author{Liangjie Cao}
\date{April 17 2018}
\maketitle
\par
Today I continue to read << Frontier Tech >> . I pick up this article << Project Loon by Google >> . It says that many of us take the Internet for granted , but large part of the world are still unconnected . Because the further the distance is , the fewer people are until eventually connectivity just stops . That's why we want to bringing the benefits of the Internet to mote people , much larger areas .  \\
\par
How do we get the position ? We use big ballons filled with lighter - than - air gas and released up into the skies . The ballons can get the signal from ground stations connection to local Internet service providers . When it is eventually to come down , they can navigate to remote areas . \\
\par
In my view , Internet is a part of our daily life nowadays . By exchanging message from person to person , we can get useful information easily . Thi project makes the the truth to connect more people in more places around the world . How amazing the plan is ! \\
\footnote{\centering 英语每日练习,来源<<Frontier tech>>}
\end{document}
