\documentclass[35pt]{article}
\usepackage{ctex}
\usepackage{CJK}
\usepackage{picinpar,graphicx}
\usepackage{cite}
\usepackage{multirow}
\usepackage{hyperref,amsmath,amssymb,amscd}
\usepackage{setspace}
\setlength{\parindent}{2em}
\twocolumn
\begin{document}
\title{\textbf{The future of deep learning}}
\author{\textbf{Liangjie Cao}}
\date{\textbf{15 May 2018}}
\maketitle
\par
\setlength{\baselineskip}{30pt}
\textbf{After a long period of practicing. Maybe I should have a little feeling. Unsupervised learning had a catalytic effect in reviving interest in deep learning, but has since been overshadowed by the successes of purely supervised learning. Although we have not focused on it in this Review, we expect unsupervised learning to become far more important in the longer term. Human and animal learning is largely unsupervised: we discover the structure of the world by observing it, not by being told the name of every object. It sounds like the acknowledgement of machine. }
\par
\textbf{And the human vision, is an active process that sequentially samples the optic array in an intelligent, task-specific way using a small, high-resolution fovea with a large, low-resolution surround. We expect much of the future progress in vision to come from systems that are trained end-toend and combine ConvNets with RNNs that use reinforcement learning to decide where to look. Systems combining deep learning and reinforcement learning are in their infancy, but they already outperform passive vision systems at classification tasks and produce impressive results in learning to play many different video games(Figure. ~\ref{Figure2}). \cite{name1}}
\par
\textbf{Natural language understanding is another area in which deep learning is poised to make a large impact over the next few years. We expect systems that use RNNs to understand sentences or whole documents will become much better when they learn strategies for selectively attending to one part at a time. Ultimately, major progress in artificial intelligence will come about through systems that combine representation learning(Table.~\ref{Table1}) with complex reasoning. Although deep learning and simple reasoning have been used for speech and handwriting recognition for a long time, new paradigms are needed to replace rule-based manipulation of symbolic expressions by operations on large vectors. I think the future is brighter and brighter.(Figure.~\ref{Figure1})}\\
\newpage
\onecolumn
 \begin{figure}[htbp]
 \centering
 \includegraphics[width=0.5\textwidth]{b.png}\\
 \caption{Deep Learning}\label{Figure1}
  \centering
 \includegraphics[width=0.5\textwidth]{videogame.png}\\
 \caption{Typical video game}\label{Figure2}
\end{figure}
 \begin{table}[!htbp]
  \centering
 \begin{tabular}{|p{3cm}|p{3cm}|p{3cm}|}
    \hline
      & Typical learning & Deep learning\\
    \hline
    Leaning method & Feature Engineering & representation learning\\
    \hline
  \end{tabular}
  \caption{\textbf{Diffrences of the methods}} \label{Table1}
  \end{table}
  \bibliographystyle{plain}
\bibliography{yinyong1}
\end{document}

