\documentclass[50pt]{article}
\usepackage{ctex}
\usepackage{CJK}
\usepackage{picinpar,graphicx}
\usepackage{cite}
\usepackage{multirow}
\usepackage{hyperref,amsmath,amssymb,amscd}
\setlength{\parindent}{2em}
\begin{document}
\title{Supervised learning I}
\author{Liangjie Cao}
\date{27 April 2018}
\maketitle
\par
This paper said that supervised learning , whether deep or not , is the most common form of machine learning .\ref{Table} This means they build a system that can classify images as containing . The professors first collect a large data set of images of homes , cars , people and pets , each labelled with its category before training . \\
\par
The process is complex , because they want to have the highest score of all acategories , but this is impossible . They compute an objective function that measures the error (or distance) between the output scores and the desired pattern of scores.\cite{name1} The machine then modifies the internal adjustable pattern called weight .
 \begin{figure}[ht]
 \centering
 \includegraphics[width=5cm]{moxing.png}\\
 \caption{Supervised learning}
\end{figure}\\
\par
The main question is , how to properly adjust the weight vector for each weight which indicates by what amount the error would increase or decrease if the weight were increased by a tiny amount. I will continue learning . \\
 \begin{table}[!htbp]
  \centering
 \begin{tabular}{|c|c|c|}
  \hline
   1 & Latex1\\
  \hline
    2 & Latex2\\
  \hline
  \end{tabular}
  \caption{Classification of machine learning}\label{Table}
  \end{table}
\bibliographystyle{plain}
\bibliography{yinyong1}
\end{document}

