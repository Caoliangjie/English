\documentclass[25pt]{article}
\usepackage{ctex}
\usepackage{CJK}
\usepackage{picinpar,graphicx}
\usepackage{cite}
\usepackage{multirow}
\usepackage{hyperref,amsmath,amssymb,amscd}
\setlength{\parindent}{2em}
\twocolumn
\begin{document}
\title{\textbf{Supervised learning IV}}
\author{\textbf{Liangjie Cao}}
\date{1 May 2018}
\maketitle
\par
\textbf{Today I learn what's the key advantage of deep learning. The paper says the conventional option is to hand design good feature extractors, which requires a considerable amount of engineering skill and domain expertise. But this can all be avoided if good features can be learned automatically using a general-purpose learning procedure.\cite{name1}}\\
\par
\textbf{A deep-learning architecture is a multilayer stack of simple modules, all (or most) of which are subject to learning, and many of which compute non-linear input–to-output mappings. We can see the contrete process from the picture Figure1.\ref{Figure} It's a convolutional network. I learn that the weights of convolution neural network training can be applied to different positions of images, that is to say, the left and right images are not counted separately. The neural network trained with sparse data also achieves good results in image classification, which reduces the number of units active simultaneously in the whole model and can train non - Gaussian distribution models from it.}
\\
\par
\textbf{Actually deep-learning architecture is a multilayer stack of simple modules, it says all (or most) of which are subject to learning, and many of which compute non-linear input–output mappings.\ref{Table} After the precipitation of many related concepts, in-depth learning is gradually improved, in-depth understanding of in-depth learning is the accumulation of knowledge in all aspects. It's a long process, I will continue learning. }\\
\onecolumn
\begin{table}[!htbp]
  \centering
 \begin{tabular}{|p{3cm}|p{4cm}|p{3cm}}
   \hline
     deep-learning architecture &  multilayer stack of simple modules\\
  \hline
     multilayer stack &  Learning\\
   \hline
  \end{tabular}
  \caption{\textbf{Contrete learning purpose}} \label{Table}
  \end{table}
 \begin{figure}[ht]
 \centering
 \includegraphics[width=15cm]{dog.png}\\
 \caption{\textbf{Inside a convolutional network}}\label{Figure}
\end{figure}
\bibliographystyle{plain}
\bibliography{yinyong1}
\end{document}

