\documentclass[10pt,twocolumn,letterpaper]{article}
\usepackage{cvpr}
\usepackage{times}
\usepackage{epsfig}
\usepackage{url}
\usepackage{graphicx}
\usepackage{amsmath}
\usepackage{amssymb}
\usepackage[breaklinks=true,bookmarks=false,backref=page]{hyperref}
\cvprfinalcopy
\def\cvprPaperID{****}
\def\httilde{\mbox{\tt\raisebox{-.5ex}{\symbol{126}}}}
\begin{document}
\title{\textbf{Non-local Neural Networks}}
\author{Liangjie Cao\\\\ Aug. 15, 2018}
\maketitle
\section{Non-local network advantage}
In deep neural networks, capturing long-term dependencies is critical. For continuous data (such as speech-speaking languages), loop operations are the primary solution to long-term dependency problems in the time domain. For image data, long-distance dependencies are modeled by large receptive fields formed by a large number of convolution operations.

Convolution operations or loop operations are all processing local neighborhoods in space or time. In this way, long-distance dependencies can only be captured when these operations are applied repeatedly, and the signal can be continuously transmitted through the data. Repeated local operations have some limitations: first, the computational efficiency is low; second, the optimization difficulty is increased; finally, these challenges lead to multi-hop dependency modeling, for example, when messages need to be passed back and forth between long distances, it is very difficult .

In this paper, they propose non-local operations as an efficient, simple, and versatile component, and use deep neural networks to capture long-range dependencies. The non-local operations we propose are inspired by the general meaning of classical non-local operations in computer vision. Intuitively, the calculated response of a non-local operation at a location is a weighted sum of features at all locations in the input profile (Figure 1). A set of locations can be in space, time, or time and space, suggesting that our operations can be applied to image, sequence, and video problems.
\section{ Non-local Neural Networks}
A non-local operation~\ref{Figure1} is a flexible building block and can
be easily used together with convolutional/recurrent layers.
It can be added into the earlier part of deep neural networks,
unlike
fc
layers that are often used in the end. This allows us
to build a richer hierarchy that combines both non-local and
local information.
  \begin{figure}[!htb]
  	\centering
  	\includegraphics[width=3in]{non.png}\\
  	\caption{Non-local network}\label{Figure1}
  \end{figure}
\section{Details}
  \begin{figure}[!htb]
  	\centering
  	\includegraphics[width=3in]{it.png}\\
  	\caption{Iterations}\label{Figure2}
  \end{figure}
Our models are pre-trained on ImageNet~\cite{name35}.Unless specified, we fine-tune our models using 32-frame input clips. These clips are formed by randomly cropping out 64 consecutive frames from the original full-length video and then dropping every other frame. The spatial size is 224×224
pixels, randomly cropped from a scaled video whose shorter
side is randomly sampled in [256, 320] pixels, following~\cite{name8}.
We train on an 8-GPU machine and each GPU has 8 clips in a
mini-batch (so in total with a mini-batch size of 64 clips). We
train our models for 400k iterations in total, starting with a
learning rate of 0.01 and reducing it by a factor of 10 at every
150k iterations (see also Figure~\ref{Figure2}). We use a momentum
of 0.9 and a weight decay of 0.0001. They adopt dropout after the global pooling layer, with a dropout ratio of
0.5. We fine-tune our models with BatchNorm (BN)
enabled when it is applied. This is in contrast to common
practice [20] of fine-tuning ResNets, where BN was frozen.
We have found that enabling BN in our application reduces
overfitting.

\bibliographystyle{abbrv}
\bibliography{yinyong20}
\end{document}
