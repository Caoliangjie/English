\documentclass[50pt]{article}
\usepackage{ctex}
\usepackage{CJK}
\usepackage{picinpar,graphicx}
\usepackage{cite}
\setlength{\parindent}{2em}
\begin{document}
\title{What's Deep Learning}
\author{Liangjie Cao}
\date{April 25 2018}
\maketitle
\par
Recently I feel that I should try to read some professional articles . I ' m interested in Deep Learning so I find a paper from < < Nature > > called Deep Learning . \\
\par
This paper said that conventional machine learning techniques were limited in their ability to process natural data . Machine learning systems are used to identify objects in images , transcribe speech into text , match news items , posts or products with users interests , and select relevant results of search . \cite{name1} These applications make use of a class of techniques called Deep Learning .  \\
 \begin{figure}[ht]
 \centering
 \includegraphics[width=5cm]{laotou.png}\\
 \caption{Geoffrey Hinton}
\end{figure}
\par
They try to build a new type of frameworks in this paper , so it can easily take advantages of increases in the amount of availiable computation and data . I will continue learning how it works these days . \\
\bibliographystyle{plain}%
\bibliography{yinyong1}
\end{document}

