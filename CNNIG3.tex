\documentclass[10pt,twocolumn,letterpaper]{article}
\usepackage{cvpr}
\usepackage{times}
\usepackage{epsfig}
\usepackage{graphicx}
\usepackage{amsmath}
\usepackage{amssymb}
\usepackage[breaklinks=true,bookmarks=false,backref=page]{hyperref}
\cvprfinalcopy
\def\cvprPaperID{****}
\def\httilde{\mbox{\tt\raisebox{-.5ex}{\symbol{126}}}}
\begin{document}
\title{\textbf{Image Question Answering using Convolutional Neural Network with Dynamic Parameter Prediction II
}}
\author{Liangjie Cao\\\\ Jun 26, 2018}
\maketitle
\section{Parameter Hashing}
The weights in the dynamic parameter layers are determined based on the learned model in the parameter prediction network given a question. The most straightforward
approach to obtain the weights is to generate the whole matrix $W_d(q)$ using the parameter prediction network. However, the size of the matrix is very large, and the network may be overfitted easily given the limited number of train-
ing examples. In addition, since we need quadratically more
parameters between GRU and the fully-connected layer in
the parameter prediction network to increase the dimensionality of its output, it is not desirable to predict full weight matrix using the network. Therefore, it is preferable to construct $W_d(q)$ based on a small number of candidate weights
using a hashing trick. The authors employ the recently proposed random weight sharing
technique based on hashing~\cite{name3} to construct the weights in
the dynamic parameter layer. Specifically, a single param-
eter in the candidate weight vector $p$ is shared by multiple
elements of $W_d(q)$, which is done by applying a predefined
hash function that converts the 2D location in $W_d(q)$ to the
1$D$ index in $p$ By this simple hashing trick, they can reduce the number of parameters in $W_d(q)$ while maintaining the
accuracy of the network~\cite{name3}.
\par Let $\omega_{mn}^d$ be the element at (m, n) in $W_d(q)$, which corresponds to the weight between $m_th$ output and $n_th$ input
neuron. Denote by $\psi(m, n)$ a hash function mapping a key
(m, n) to a natural number in {1, . . . , $K$}, where $K$ is the
dimensionality of $p$. The final hash function is given by
\begin{equation}
\omega_{mn}^d=p_\psi(m,n)\cdot\xi(m,n)
\end{equation}
where $\xi(m, n)$ : N × N $\to$ \{+1, −1\} is another hash function independent of $\psi(m, n)$. This function is useful to remove the bias of hashed inner product~\cite{name3}.	
\par The authors believe that it is reasonable to reduce the number of free parameters based on the hashing technique as there are many redundant parameters in deep neural networks and the network can be parametrized using a smaller set of candidate weights. Instead of training a huge number of parameters without any constraint, it would be advantageous practically to allow multiple elements in the weight matrix to share the same value. It is also demonstrated that thennumber of free parameter can be reduced substantially with little loss of network performance~\cite{name3}.
\begin{figure}[!htb]
 \centering
 \includegraphics[width=0.4\textwidth]{comp.png}\\
 \caption{Comparison of GRU and LSTM. Contrary to LSTM that
contains memory cell, GRU updates the hidden state directly}\label{Figure1}
 \end{figure}
 \par As illustrated in Figure~\ref{Figure1}, such dependency is captured by
adaptively updating its hidden states with gate units. How-
ever, contrary to LSTM, which maintains a separate mem-
ory cell explicitly, GRU directly updates its hidden states
with a reset gate and an update gate. The detailed proce-
dure of the update is described below.
\section{Training Algorithm}
To deal with the deficiency of linguistic information in
ImageQA problem, we transfer the information acquired
from a large language corpus by fine-tuning the pre-trained
embedding network. We initialize the GRU with the skip-
thought vector model trained on a book-collection corpus
containing more than 74M sentences~\cite{name14}. Note that the
GRU of the skip-thought vector model is trained in an un-
supervised manner by predicting the surrounding sentences
from the embedded sentences. As this task requires to un-
derstand context, the pre-trained model produces a generic
sentence embedding, which is difficult to be trained with
a limited number of training examples. By fine-tuning our
GRU initialized with a generic sentence embedding model
for ImageQA, we obtain the representations for questions
that are generalized better.
\section{Fine-tuning CNN}
It is very common to transfer CNNs for new tasks in
classification problems, but it is not trivial to fine-tune the
CNN in our problem. They observe that the gradients below
the dynamic parameter layer in the CNN are noisy since its
weights are predicted by the parameter prediction network.
Hence, a naı̈ve approach to fine-tune the CNN typically fails to improve performance, and they employ a slightly different technique for CNN fine-tuning to sidestep the observed problem. We update the parameters of the network using
new datasets except the part transferred from VGG 16-layer
net at the beginning, and start to update the weights in the
subnetwork if the validation accuracy is saturated.
\section{Evaluation Metrics}
DAQUAR and COCO-QA employ both classification ac-
curacy and its relaxed version based on word similarity,
WUPS~\cite{name17}. It uses thresholded Wu-Palmer similarity
based on WordNet taxonomy to compute the similarity
between words. For predicted answer set $A^i$ and ground-
truth answer set $T^i$ of the $i^{th}$ example, WUPS is given by
\begin{equation}
WUPS=\frac{1}{N}\sum_{i=1}^{N}\min\{\prod_{a\in{A^i}}\max_{t\in{T_i}}\mu(a,t),\prod_{t\in{T^i}}\max_{a\in{A_i}}\mu(a,t)\}
\end{equation}
where $\mu(\cdot, \cdot)$ denotes the thresholded Wu-Palmer similarity
between prediction and ground-truth. The authors use two threshold
values (0.9 and 0.0) in our evaluation.
\bibliographystyle{abbrv}
\bibliography{yinyong6}
\end{document}

